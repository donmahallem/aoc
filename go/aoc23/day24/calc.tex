\documentclass{article}
\usepackage{amsmath}
\usepackage{amssymb}
\usepackage{mathtools}
\usepackage{geometry}
\geometry{a4paper}

\title{Advent of Code 2023 - Day 24 Solution - Part 2}
\date{}

\begin{document}

\maketitle

\section*{Definitions}

\begin{align*}
    i, j &\in \mathbb{N} & \text{indices of specific hailstones} \\
    \vec{r}_s &= [r_{sx}, r_{sy}, r_{sz}]^T & \text{stone initial position (unknown)} \\
    \vec{v}_s &= [v_{sx}, v_{sy}, v_{sz}]^T & \text{stone velocity (unknown)} \\
    \vec{r}_i, \vec{v}_i & & \text{position and velocity of hailstone } i \\
    t_i & & \text{collision time with hailstone } i
\end{align*}

\section*{The Collinearity Constraint}

For the stone to collide with hailstone $i$, they must occupy the same point in space at time $t_i$:
\begin{align}
    \vec{r}_s + t_i \vec{v}_s = \vec{r}_i + t_i \vec{v}_i
\end{align}
Rearranging to group terms involving time:
\begin{align}
    \vec{r}_s - \vec{r}_i = t_i (\vec{v}_i - \vec{v}_s)
\end{align}
This equation implies that the vector difference in positions $(\vec{r}_s - \vec{r}_i)$ is parallel to the vector difference in velocities $(\vec{v}_i - \vec{v}_s)$, as one is a scalar multiple ($t_i$) of the other.

\section*{Linearization via Cross Product}

Since the two vectors are parallel, their cross product must be zero:
\begin{align}
    (\vec{r}_s - \vec{r}_i) \times (\vec{v}_i - \vec{v}_s) = \vec{0}
\end{align}
Expanding this cross product:
\begin{align}
    (\vec{r}_s \times \vec{v}_i) - (\vec{r}_s \times \vec{v}_s) - (\vec{r}_i \times \vec{v}_i) + (\vec{r}_i \times \vec{v}_s) = \vec{0}
\end{align}
Rearranging to isolate the non-linear term $(\vec{r}_s \times \vec{v}_s)$:
\begin{align} \label{eq:cross}
    \vec{r}_s \times \vec{v}_s = (\vec{r}_s \times \vec{v}_i) + (\vec{r}_i \times \vec{v}_s) - (\vec{r}_i \times \vec{v}_i)
\end{align}
The term $\vec{r}_s \times \vec{v}_s$ contains the product of two unknowns. However, this term is identical for \textbf{every} hailstone. We can eliminate it by considering a second hailstone $j$ and subtracting the two equations.

Using equation (\ref{eq:cross}) for hailstone $j$:
\begin{align}
    \vec{r}_s \times \vec{v}_s = (\vec{r}_s \times \vec{v}_j) + (\vec{r}_j \times \vec{v}_s) - (\vec{r}_j \times \vec{v}_j)
\end{align}
Equating the RHS for $i$ and $j$:
\begin{align*}
    (\vec{r}_s \times \vec{v}_i) + (\vec{r}_i \times \vec{v}_s) - (\vec{r}_i \times \vec{v}_i) &= (\vec{r}_s \times \vec{v}_j) + (\vec{r}_j \times \vec{v}_s) - (\vec{r}_j \times \vec{v}_j)
\end{align*}
Grouping unknowns ($\vec{r}_s, \vec{v}_s$) on the left and knowns on the right:
\begin{align}
    \vec{r}_s \times (\vec{v}_i - \vec{v}_j) + \vec{v}_s \times (\vec{r}_j - \vec{r}_i) = (\vec{r}_i \times \vec{v}_i) - (\vec{r}_j \times \vec{v}_j)
\end{align}
This is now a \textbf{linear vector equation}. Since it represents a cross product in 3D, it provides 3 scalar linear equations for every pair of hailstones.

\section*{System Construction}

Let the unknowns be the state vector $\mathbf{x} = [r_{sx}, r_{sy}, r_{sz}, v_{sx}, v_{sy}, v_{sz}]^T$.
We use two pairs of hailstones (e.g., $1 \to 2$ and $1 \to 3$) to generate 6 equations for our 6 unknowns.

Let $\Delta \vec{v}_{ij} = \vec{v}_i - \vec{v}_j$ and $\Delta \vec{r}_{ji} = \vec{r}_j - \vec{r}_i$.
The equation is:
\[
\vec{r}_s \times \Delta \vec{v}_{ij} + \vec{v}_s \times \Delta \vec{r}_{ji} = \vec{C}_{ij}
\]
where $\vec{C}_{ij} = (\vec{r}_i \times \vec{v}_i) - (\vec{r}_j \times \vec{v}_j)$.

\subsection*{Matrix Form}
The cross product $\vec{a} \times \vec{b}$ can be written as matrix multiplication $[\vec{a}]_\times \vec{b}$, where $[\vec{a}]_\times$ is the skew-symmetric matrix:
\[
[\vec{a}]_\times = 
\begin{bmatrix}
0 & -a_z & a_y \\
a_z & 0 & -a_x \\
-a_y & a_x & 0
\end{bmatrix}
\]
Note that $\vec{r}_s \times \Delta \vec{v}_{ij} = -\Delta \vec{v}_{ij} \times \vec{r}_s = -[\Delta \vec{v}_{ij}]_\times \vec{r}_s = [\Delta \vec{v}_{ji}]_\times \vec{r}_s$.

Therefore, for a pair of hailstones $i, j$, we obtain a $3 \times 6$ block of the matrix $A$:
\[
\begin{bmatrix}
    [\vec{v}_j - \vec{v}_i]_\times & [\vec{r}_j - \vec{r}_i]_\times
\end{bmatrix}
\begin{bmatrix}
    \vec{r}_s \\ \vec{v}_s
\end{bmatrix}
= 
(\vec{r}_i \times \vec{v}_i) - (\vec{r}_j \times \vec{v}_j)
\]

\section*{Explicit Matrix $A$}

To solve for the 6 unknowns, we stack the blocks for pair $(1,2)$ and pair $(1,3)$.

Let $\Delta v_{x} = v_{2x} - v_{1x}$, $\Delta r_{x} = r_{2x} - r_{1x}$, etc.
The first 3 rows (derived from pair 1, 2) are:

\[
\begin{bmatrix}
    0 & -\Delta v_z & \Delta v_y & 0 & -\Delta r_z & \Delta r_y \\
    \Delta v_z & 0 & -\Delta v_x & \Delta r_z & 0 & -\Delta r_x \\
    -\Delta v_y & \Delta v_x & 0 & -\Delta r_y & \Delta r_x & 0
\end{bmatrix}
\begin{bmatrix}
    r_{sx} \\ r_{sy} \\ r_{sz} \\ v_{sx} \\ v_{sy} \\ v_{sz}
\end{bmatrix}
= 
\vec{b}_{12}
\]
\textit{(Note the sign flip in the $\Delta v$ block because $\vec{r}_s$ was the first term in the cross product).}

Repeating this for pair $(1,3)$ gives the bottom 3 rows. The resulting system $Ax = b$ can be solved via Gaussian elimination or standard linear solver libraries.

\end{document}